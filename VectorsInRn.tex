\documentclass[Main.tex]{subfiles} 
\begin{document}
\part{Vectors, Matrices and Systems of Equations}
\section{Vectors in $\R^n$}

\subsection{Motivation}

By now, you probably feel pretty comfortable doing mathematics with a single variable.
You can add, subtract, multiply, divide, calculate trigonometric functions and even take derivatives an integrals.  
You are probably quite used to the situation when you have one ``independent variable'', perhaps called $x$, and one ``dependent variable'', perhaps called $y=f(x)$.  
Perhaps you can even throw in a $z$ or a $w$ without panicking, but, unfortunately, many (if not most) problems arising naturally in every branch of science and engineering will have hundreds or thousands or millions of variables.  
One quickly becomes bogged down in complexity.
We need a way to organize all this data which hopefully enlightens, informs and even inspires us.
This will be perhaps the primary goal of a first course in linear algebra.  

Back in the single variable days, we would say $x$ is a real number, of $x\in \R$, and visualize $x$ as living on the ``number line'' (which we will call $\R^1$). 
We might consider a pair of real numbers $(x,y)$, and draw it as a point on a plane ($\R^2$), or a triple $(x,y,z)$ and draw it as a point in space ($\R^3$).  
Obviously, we can draw these points on a plane or in space \emph{even} if they don't represent an actual location; say if we want to plot money vs time, 
It still helps to think about it ``geometrically''.  
The brilliant-yet-simple observation is this:  
while one cannot so easily draw the point $(x,y,z,w)$, there is absolutely nothing stopping us from writing down a 4-tuple, or a 5-tuple, or a 34-tuple, or a 3511-tuple.

\begin{Def}[Vectors and the Vector Space $\R^n$]
  Define $\R^n$ to be the set of $n$-tuples of real numbers $(x_1,x_2,...,x_n)$.  
  We will often refer to $\R^n$ as a \emph{vector space} and call the tuples in $\R^n$ \emph{vectors}.  
  We will denote vectors by lower-case letters with arrows above them, like $\vec{x}=(x_1,x_2,...,x_n)$.  
  We will sometimes call a real number a \emph{scalar}, to distinguish it from a vector.  
  We will call the numbers in a vector the \emph{coordinates}.  
\end{Def}


\begin{Remark}
  Notice I did not mention ``magnitude and direction''\footnote{Like the nemesis from Despicable Me}.  
  An arrow with a certain length pointing in a certain direction is a great image to have in your head, 
  but you might have trouble saying what you mean by ``direction'' when $n>3$.  
  Still, if the mnemonic might helps you visualize the situation at hand, then all the better.
\end{Remark}

\begin{Remark}
  There isn't much of a difference between a scalar $x\in \R$ and a 1-dimensional vector $\vec{x}=(x_1)\in \R^1$.  
  While it does not much matter, we will call it a vector $\vec{x}$ when we want to emphasize vector-like properties and a scalar $x$ otherwise.  
  When trying to understand $\R^n$ in general, always keep in mind what happens in $\R^1$.  The situation will almost always be extremely simple, but possibly enlightening.  
\end{Remark}

\begin{Remark}
  Some people will tell you there's a difference between points and vectors.  We will not make that distinction in these notes.  
\end{Remark}

\begin{UnimportantRemark}
  What is $\R^0$?  
  If $\R^1$ is a line and $\R^2$ is a plane, then $\R^0$ should be a point.  
  Therefor we say there is but one vector in $\R^0$, and we'll call it $\vec{0}$.  
  Of course, this will never be important, so if you feel disturbed by 0-tuples, feel free to ignore $\R^0$.  
\end{UnimportantRemark}

\subsection{Addition and Subtraction}

We define addition of vectors as one might expect.
Let $\vec{x}$ and $\vec{y}$ be two vectors in $\R^n$, that is,
\[\vec{x}=\vect{x_1 \\ \vdots \\ x_n}\]
\[\vec{y}=\vect{y_1 \\ \vdots \\ y_n}\]
Then we define
\[\vec{x}+\vec{y}=\vect{x_1+y_1 \\ \vdots \\ x_n+y_n}\]
that is, we just add vectors ``coordinate by coordinate''.
Thus, in $\R^3$, 
\[\vect{1 \\ 2 \\ 3} + \vect{4 \\ 5 \\ 6} = \vect{5\\7\\ 9}\]
Subtraction is done in much the same way:
\[\vec{x}-\vec{y}=\vect{x_1-y_1 \\ \vdots \\ x_n-y_n}\]
so
\[\vect{4 \\ 3 \\ 2} + \vect{1 \\ 2 \\ 3} = \vect{3\\1\\-1}\]
\begin{Ex}
  Interpret addition of vectors in $\R^2$ geometrically.  
  If $\vec{x}$ and $\vec{y}$ are two points in the plane, can you come up with a geometric rule for the position of $\vec{x}+\vec{y}$?
  What about $\vec{x}-\vec{y}$?  
  (hint: what happens when you connect the dots $(0,0)$, $\vec{x}$, $\vec{y}$ and $\vec{x}+\vec{y}$?)
\end{Ex}
\begin{Ex}
  \label{sec:abgroup-laws}

  Convince yourself that vector addition is ``commutative'', that is $\vec{x}+\vec{y}=\vec{y}+\vec{x}$. 
  Then convince yourself it is ``associate'', that is $(\vec{x}+\vec{y})+\vec{z}=\vec{x}+(\vec{y}+\vec{z})$. 
  If you're unsure of how to begin, write out the right side and left side of each equation and notice they are the same.  
  Use the fact that scalar addition is commutative and associative.  
\end{Ex}

Let $\vec{0}$ be the vector with zero in each coordinate, that is 
\[\vec{0}=\vect{0 \\ \vdots \\ 0}\]
\begin{EasyEx}
  Let $\vec{x}\in \R^n$ be any vector.  Show $\vec{x}+\vec{0} = \vec{x}$.  
  Show that $\vec{x}-\vec{x}=\vec{0}$.
\end{EasyEx}
We see that addition and subtraction of vectors obey some laws we are already used to.
Vector addition is just an extension of scalar addition; we're just adding and subtracting in bulk!


\subsection{Scalar Multiplication}
There is nothing stopping us from defining multiplication or division ``coordinate by coordinate'', but we will not do this, because we do not need to.  
While we don't allow multiplying two vectors, we do allow multiplying a scalar by a vector.  
Let $c\in \R$ be a scalar and $\vec{x}\in\R^n$.  
Define $c\cdot \vec{x}$ by
\[c\cdot\vec{x}=\vect{c\cdot x_1 \\ \vdots \\ c\cdot x_n}\]
\begin{ImpEasyEx}
  Interpret scalar multiplication of vectors in $\R^2$ geometrically.  
  If $\vec{x}$ is a vector and $c$ a scalar, where is $c\cdot\vec{x}$?
  (hint: do $c>0$, $c=0$ and $c<0$ separately).  
\end{ImpEasyEx}
\begin{ImportantRemark}
  In the past, everything in sight has been a number, so one didn't need to worry about mixing apples and oranges.
  In linear algebra, there one must keep straight many different fruits: infinitely fruits many in fact!
  One cannot add a vector and a scalar, nor can you add a vector in $\R^n$ with a vector in $\R^m$ if $n\ne m$.  
  A vector plus a vector is a vector of the same size.  
  One cannot multiply vectors, but one can multiply a vector times a scalar and get a new vector of the same size.
  The following easy exercise should give you a chance to practice keeping track of types.  
\end{ImportantRemark}
\begin{ImpEasyEx}
  \label{sec:module-laws}
  The following laws should be easy to show.
  \begin{enumerate}[a)]
  \item $0\cdot\vec{x} = \vec{0}$
  \item $(a+b)\cdot\vec{x} = a\cdot\vec{x} + b\cdot\vec{x}$
  \item $a\cdot (\vec{x} + \vec{y}) = a\cdot\vec{x} + a\cdot\vec{y}$
  \item $(a\cdot b)\cdot\vec{x} = a\cdot (b\cdot\vec{x})$
  \end{enumerate}
\end{ImpEasyEx}

\begin{RemarkExp}
  It turns out that \ref{sec:abgroup-laws} and \ref{sec:module-laws} are actually the only properties one really needs to do linear algebra.
  Rather than only considering $\R^n$, we could consider ``any object with these properties'', and much of what we will say will come out exactly the same.
  We choose to consider only $\R^n$ for the sake of concreteness.  
\end{RemarkExp}



\section{Spans of Vectors}

Let us begin with the definition.
\begin{Def}
  Let $\vec{v_1},\cdots,\vec{v_k}$ be vectors in $\R^n$.
  Then we say another vector $\vec{x}\in\R^n$ is in the span of $\vec{v_1},\cdots,\vec{v_k}$, written $\vec{x}\in\spanv{\vec{v_1},\cdots,\vec{v_k}}$, if there are $k$ scalars $c_1,\cdots, c_k$ such that 
  \[\vec{x} = c_1\vec{v_1}+\cdots+c_k\vec{v_k}\]
\end{Def}

This is a rather abstract definition, so lets unpack it a bit.
\begin{EasyEx}
  \label{sec:spanexp}
  Let $\vec{x}\ne \vec{0}$ be any vector in $\R^n$.  Describe
  \begin{enumerate}[a)]
  \item $\spanv{\vec{x}}$
  \item $\spanv{\vec{x},2\vec{x}}$
  \item $\spanv{\vec{0}}$
  \item If $\vec{x},\vec{y}$ are two vectors in $\R^3$ that are not multiples of each other, describe $\spanv{\vec{x},\vec{y}}$.  
  \end{enumerate}
\end{EasyEx}

\begin{EasyEx}
  Show that for \emph{any} $\vec{x_1},\cdots,\vec{x}_k$, $\vec{0}\in\spanv{\vec{x_1},\cdots,\vec{x_k}}$
\end{EasyEx}



\section{Linear Independence}

\begin{Def}
  We say that a set of vectors $\vec{v_1},\cdots,\vec{v_k}\in\R^n$ is linearly dependent if there is a set of scalars $c_1,...,c_k$, \emph{at least one nonzero}, such that
  \[c_1\vec{v_1}+\cdots+c_k\vec{v_k} = \vec{0}\]
  A set is linearly independent if it is not linearly dependent.  
\end{Def}

\begin{Remark}
  \label{sec:redundant}
  If a set of vectors are linearly dependent, at least one of the vectors is ``redundant''; you could throw out a vector without changing the span.  
  Do you see why this is?
\end{Remark}

\begin{UnimpEx}
  If one of the $v_i=\vec{0}$, is $\vec{x_1},\cdots,\vec{x_k}$ linearly independent?  
\end{UnimpEx}

\begin{Ex}
  Show that the span of any two vectors in $\R^1$ are linearly dependent.
  What about three vectors in $\R^2$?
  What about $n+1$ vectors in $\R^n$ (trickier).  
\end{Ex}

\begin{ImpEx}
  Make \ref{sec:redundant} precise, and prove it!
  (hint: suppose $c_1\vec{v_1}+\cdots+c_k\vec{v_k} = \vec{0}$ and $c_1\ne 0$, so you can divide by it.  Now ``solve for'' $\vec{v_1}$)
\end{ImpEx}

\exersisesc

\section{Dot and Cross Products}

In this chapter, we are going to talk about some ways of measuring vectors.
The ``magnitude-and-direction'' viewpoint is especially helpful in this case.

\subsection{Dot Products}

Remember that we don't allow you to multiply two vectors and get another vector.
The dot product may be called a product and look like multiplication, but you should think of it as a sort of measurement of size and similarity.
In particular, the dot product measures how ``big'' and close together vectors are.

There are two different formulas for the dot product.
We will start with the first
\begin{Def}[Dot Product]
  Let
  \[\vec{x}=\vect{x_1\\\vdots\\x_n} \hspace{10mm}\vec{y}=\vect{u_1\\\vdots\\u_n}\]
  be two vectors in $\R^n$.
  Then define
  \[\vec{x}\cdot \vec{y} = x_1y_1 + \cdots + x_ny_n\]
\end{Def}
\begin{Remark}
  As always, we say the types.
  The dot product takes two vectors in $\R^n$ and gives back a scalar.
\end{Remark}
Lets do some easy properties of the dot product, before we go on to see how its useful.
These exercises should be as easy as writing out both sides of the equation and noticing they are the same.  
\begin{EasyEx}
  Show that, if $\vec{e}_i$ is the standard basis vector in $\R^n$ (that is, all coordinates 0 except the $i^{th}$, which is 1), and if $\vec{x}=\vect{x_1\\\vdots\\x_n}$, then
  \[\vec{e}_i\cdot \vec{x} = x_i\]
\end{EasyEx}
\begin{EasyEx}
  Show that, if $\vec{x},\vec{y}\in\R^n$, then
  \[\vec{x}\cdot\vec{y}=\vec{y}\cdot\vec{x}\]
\end{EasyEx}
\begin{EasyEx}
  \label{sec:bilinear}
  Show that, if $\vec{x},\vec{y}\in\R^n$ and $c\in \R$, then
  \[(c\vec{x})\cdot\vec{y}=c(\vec{x}\cdot\vec{y})=\vec{x}\cdot(c\vec{y})\]
  So scalars ``distribute'' in a funny little way.
\end{EasyEx}
We begin by thinking about what happens if you dot a vector in $\R^2$ with itself.
\begin{ImpEx}
  \label{sec:selfdotlen}
  Use the Pythagorean Theorem to show that if $\vect{x\\y}\in \R^2$, then 
  \[\sqrt{\vect{x\\y}\cdot\vect{x\\y}}\]
  is the ``length'' of vector, that is, the distance between the point $\vect{x\\y}$ and $\vect{0\\0}$.
  (hint: if you get stuck, draw a picture!)
  Does this still work for $\vect{x\\y\\z}\in\R^3$?
  How about $\vec{x}\in \R^{41}$?
\end{ImpEx}
Inspired by \label{sec:selfdotlen}, we define the length of a vector
\begin{Def}[Length of a Vector]
  Let $\vec{v}\in \R^n$.  
  Define the length $\vec{v}$, denoted $\|\vec{v}\|$, by
  \[\|\vec{v}\| = \sqrt{\vec{v}\cdot\vec{v}}\]
  The length is also sometimes called the ``magnitude'' or ``norm''.
\end{Def}
The next exercise makes good intuitive sense:
\begin{EasyEx}[Lengths Behave as they Ought to]
  Show that, if $\vec{x}\in\R^n$ and $c\in \R$ and $c\ge 0$, then 
  \[\|c\vec{x}\| = c\|\vec{x}\|\]
  (hint: \ref{sec:bilinear})
\end{EasyEx}
This is great.  At least when you dot a vector with itself, it tells you how ``big'' the vector is.
Even better, if you multiply a vector by 2, its ``gets twice as big''.

Now let's investigate what happens if we dot any old vectors together.
Again, we start by considering the case in $\R^2$.
\begin{ImpEx}[Cosine Dot Product Formula]
  Let $\vec{x}$ and $\vec{y}$ be vectors in $\R^2$, and $\theta$ the angle between them.
  By applying the law of cosines to the triangle with corners at $\vec{0}$, $\vec{x}$ and $\vec{y}$, show that 
  \[\vec{x}\cdot\vec{y}=\|\vec{x}\|\|\vec{y}\|\cos(\theta)\]
  (hint: the length of the third side of this triangle is $\|\vec{x}-\vec{y}\|$.  What happens when you expand that into coordinates?)
\end{ImpEx}
Once again, we are inspired by the situation in $\R^2$, so we want this to be true for any $n$.
However, it can be hard to say exactly what we mean by ``angle between two vectors in $\R^5$''.
Since we don't quite have the language to say what we mean, I will tell a somewhat fanciful story.
Take two vectors in $\R^n$ and rotate them so that they are both in the $x_1x_2$-plane, which looks kind of like $\R^2$.
Of course, I haven't said why rotating doesn't change the dot product, or even what I meant by rotating in higher dimensions\footnote{If you go on to a higher level linear algebra class like math 113, they will say, roughly, that a rotation is just anything which doesn't change the dot product}, but bear with me.  
Since the two vectors are nicely aligned with the $x_1x_2$-plane, the law of cosines works, and we get the following theorem.  
\begin{Theorem}[Dot Product Cosine Formula]
  \label{sec:cosineform}
  Let $\vec{x}$ and $\vec{y}$ be vectors in $\R^n$ (for any $n$), and $\theta$ the ``angle between'' them.
  \[\vec{x}\cdot\vec{y}=\|\vec{x}\|\|\vec{y}\|\cos(\theta)\]
\end{Theorem}
Thus, the dot product measures not only how big things are, but also how narrow the angle between them is.
If the two vectors are in the same line, the angle between them is 0, and $\cos(0)=1$, so we just get the product of the lengths.
If the two vectors are perpendicular, the angle between them is $\pi/2$, and since $\cos(\pi/2)=0$, we get the dot product is zero, no matter how big the vectors involved are.
For what its worth, if two vectors are pointing in opposite directions (that is, the angle between them is more than $\pi/2$ but less than $3\pi/2$) the sign of the formula aboove will be negative.

Since mathematicians love making up words, here is a synonym for perpendicular that will be used throughout this book:
\begin{Def}[Orthogonal]
  Vectors $\vec{x}$ and $\vec{y}\in\R^n$ are called orthogonal if
  \[\vec{x}\cdot\vec{y}=0\]
\end{Def}
\begin{EasyEx}
  Show that the $\vec{0}$ is orthogonal to every other vector.
\end{EasyEx}



\subsection{Cross Products}

The cross product is a special object which only works in $\R^3$.  
Given two vectors in $\R^3$ which are not multiples of each other, there is a unique line orthogonal to them both (try to visualize this).
The cross product will give you a vector along that line, whose length is determined not by how small the angle between the two vectors is, but by how large.
It is defined as follows
\begin{Def}[Cross Product]
  Define the cross product of two vectors in $\R^3$ by
  \[\vect{x_1\\x_2\\x_3}\times\vect{y_1\\y_2\\y_3} = \vect{x_2y_3-x_3y_2\\x_3y_1-x_1y_3\\x_1y_2-x_2y_1}\]
\end{Def}
Wow; that formula is a complicated looking!
Remembering that we care about types, note that the cross product takes two vectors in $\R^3$ and returns a vector in $\R^3$.  
Remember that the goal of the cross product is to get a vector orthogonal to both of the two original vectors.
\begin{ImpEx}
  Show that if $\vec{x},\vec{y}\in\R^3$, then $\vec{x}\times\vec{y}$ is orthogonal to both $\vec{x}$ and $\vec{y}$ by writing out the dot product explicitly and seeing that everything cancels.
\end{ImpEx}
\begin{EasyEx}
  The cross product is not commutative.
  Show that
  \[\vec{x}\times\vec{y} = - \vec{y}\times\vec{x}\]
\end{EasyEx}
\begin{EasyEx}
  Show that
  \[\vec{x}\times\vec{x}=\vec{0}\]
\end{EasyEx}
\begin{EasyEx}
  Show that, if $\vec{x},\vec{y}\in\R^3$ and $c\in \R$, then
  \[(c\vec{x})\times\vec{y}=c(\vec{x}\times\vec{y})=\vec{x}\times(c\vec{y})\]
\end{EasyEx}

We will now prove one of the more important properties of the cross product in a few steps.
\begin{Ex}[optional, but make sure you know the result]
  \label{sec:crossid}
  Show that, for $\vec{x},\vec{y}\in\R^3$, that
  \[\|\vec{x}\times\vec{y}\|^2 = \|\vec{x}\|^2\|\vec{y}\|^2 - (\vec{x}\cdot\vec{y})^2\]
  This should amount to expanding out both sides and noticing they are the same.
  Make sure you have plenty of paper for this one, this is a good bit of number crunching.
\end{Ex}
\begin{Ex}
  \label{sec:sineform}
  Use \ref{sec:crossid}, \ref{sec:cosineform} and the fact that $\sin^2(x)=1-\cos^2(x)$ to show that,
  if $\vec{x},\vec{y}$ are two vectors in $\R^3$ with angle $\theta$ between them, then
  \[\|\vec{x}\times\vec{y}\| = \|\vec{x}\|\|\vec{y}\|\sin(\theta)\]
\end{Ex}

\begin{EasyEx}
  \label{sec:crossprodisarea}
  Recalling that the area of a triangle with sides of length $a,b$ and $c$ and angle $\theta$ between the sides of length $a$ and $b$ is
  \[\frac{ab\sin(\theta)}{2}\]
  show that $\|\vec{x}\times\vec{y}\|$ is the area of the parallelogram with corners $\vec{0}$, $\vec{x}$, $\vec{y}$ and $\vec{x}+\vec{y}$.
  (hint: \ref{sec:sineform})
\end{EasyEx}

\begin{Remark}
  Cross products and dot products are geometric objects, so make sure you know when to use them.  
  The first midterm tends to have one or two cute geometry problems, which the alert student will know to solve with dot products and cross products.
  Often times you will have to compute the area or some angle of a random triangle in $\R^3$, or find a vector perpendicular to it.
  If you come across such a question, start by translating the figure so that one of the interesting corners is at the origin.
  That is, given a triangle with corners $\vec{a},\vec{b}$ and $\vec{c}$, define $\vec{v}=\vec{b}-\vec{a}$ and $\vec{w}=\vec{c}-\vec{a}$.
  The triangle with corners $\vec{0},\vec{v}$ and $\vec{w}$ is geometrically the same as the original triangle (just moved over a bit), 
  but the dot product, cross product and length formula can give you information about it.
\end{Remark}
  

\exersisesd

\section{Systems of Linear Equations}

Systems of linear equations are ubiquitous in all branches of science and engineering.
Being able to identify, understand and solve them is an invaluable tool for anyone who wants to deal with data.
Systems of linear equations will appear as a sub-problem in an extraordinary number of contexts and you must be able to deal with them when they appear.
Luckily there are tons of computer packages which can help you deal with them when they appear in real life situations... but first you need to learn the rules!
Studying linear equations--their anatomy and solutions--will be the topic of the first half of these notes.

\subsection{The Problem}

A linear equation is an equation of the form
\[m_1 x_1 + \cdots + m_n x_n - b = 0\]
where the $m_i$ and $b$ are fixed numbers and the $x_i$ are variables.  If $n=1$, this is just very familiar; 
\[mx -b = 0\]
When $n=2$, these equation are 
\[ax+by=c\]
and they implicitly define a line (I am playing fast and loose with the variable names here.  If $n$ is small, you can, if you so please, call $x_1$ by $x$ and $x_2$ by $y$).
Notice that there could be lots of solutions for $n> 1$, for instance there are infinitely many on a line.  
Notice, however, that there need not be \emph{any} solutions.  For instance, consider
\[0\cdot x + 1 = 0\]


\begin{EasyEx}
  \label{sec:dot-is-lineareq}
  Let $\vec{m},\vec{x}\in\R^n$, where $\vec{m}$ is some fixed vector and $\vec{x}$ is a variable vector. 
  Show that the equation
  \[\vec{m}\cdot\vec{x} = -b\]
  is ``the same'' as a linear equation.  (hint: vector dot vector = scalar)
\end{EasyEx}

\begin{Remark}
  The nice thing about writing a linear equation in the above way is that it emphasizes the ``geometric'' nature of linear equations.
  Sure, the original definition is strictly ``algebraic'', but considering the set of solutions as points in $\R^n$ immediately gives us an intuitive picture for what would otherwise just be a mess of numbers.  

  Temporarily, let us restrict to the case $b=0$ and $\vec{m}\ne \vec{0}$.  
  Then, by \ref{sec:dot-is-lineareq}, the set of solutions to the linear equation $\vec{m}\cdot\vec{x}=0$ is merely the set of all vectors orthogonal to $\vec{m}$.  
  If $\vec{m}\in\R^2$, vectors on the line perpendicular to $\vec{m}$ solve this equation.  
  If $\vec{m}\in\R^3$, vectors in the plane perpendicular to $\vec{m}$ solve this equation.
  This ``geometric'' intuition remains correct when $n\ge 3$.  
  When we talk about dimension, we will be able to make this precise, but in $\R^n$ the set of solutions to $\vec{m}\cdot\vec{x}=0$ should be somehow an $(n-1)$-dimensional ``plane''.  
\end{Remark}

If we had two vectors, $\vec{m}_1$ and $\vec{m}_2$, we could consider the set of all vectors which \emph{simultaneously} solved $\vec{m}_1\cdot\vec{x}=b_1$ and $\vec{m}_2\cdot\vec{x}=b_2$.  
\begin{Ex}
  Let $n=3$ and $b_1=b_2=0$.  Describe, geometrically, the set of vectors $\vec{x}$ with $\vec{m}_1\cdot\vec{x}=\vec{m}_2\cdot\vec{x}=0$, if  
  \begin{enumerate}[a)]
  \item $\vec{m}_1$ and $\vec{m}_2$ are linearly independent.
  \item $\vec{m}_1$ and $\vec{m}_2$ are linearly dependent, but both nonzero.  
  \end{enumerate}
\end{Ex}

\begin{Def}[System of Linear Equations]
  \label{sec:systemdef}
  A system of linear $m$ linear equations in $n$ variables is a list of equations
  \[\begin{array}{ccccccccc}
    a_{1,1}x_1 & + & \cdots & + & a_{1,n}x_n & - & b_1 & = 0\\
    \vdots     &   & \ddots &   & \vdots     &   & \vdots\\
    a_{m,1}x_1 & + & \cdots & + & a_{m,n}x_n & - & b_m & = 0
  \end{array}\]
  where the $a_{i,j}\R$ are constants and the $x_i$ are variables.
  A system is called homogeneous if all the $b_i=0$, and inhomogeneous otherwise.  
  A system is inconsistent if there are no solutions.  
\end{Def}

\begin{Remark}
  \label{sec:confusing}
  We changed the name of the coefficients from $m$ to $a$ with a full system of linear equations because, traditionally, 
  $m$ is the number of equations and $n$ is the number of variables.  
  This convention is horribly confusing because, not only do the letters in question have little mnemonic meaning, but they rhyme!  
  Beware of confusing the two.  
\end{Remark}

\begin{ImpEx}
  Come up with an example of a system of linear equations with
  \begin{enumerate}[a)]
  \item exactly one solution.
  \item more than one solutions.
  \item no solutions
  \end{enumerate}
  (hint: these can all be done with a single equation and a single variable, that is, $m=n=1$)
\end{ImpEx}

\begin{EasyEx}
  Extend \ref{sec:dot-is-lineareq} to a system of $m$ linear equations.  
\end{EasyEx}

\begin{EasyEx}
  Show a homogeneous system of linear equations has at least one solution.  (hint: it should feel like cheating)
\end{EasyEx}

\begin{Ex}
  If you add an additional equation to a system of $m$ equations in $n$ variables, show that the solution set either stays the same or gets smaller (hint: a solution to the new system of $m+1$ equations is also a solution to the first $m$ equations).  
  In the case of a homogeneous system, can you come up with a rule telling if the solution set will shrink or stay the same? (hint: write system as in \ref{sec:dot-is-lineareq} and consider the span of the first $m$ coefficient vectors $\vec{a_i}$)
\end{Ex}


\subsection{The Solution}

It turns out that there is a single, end-all solution to this problem. 
First of all, you could try using the substitution method.  
The best way of seeing why this is inefficient is trying it.
\begin{UnimpEx}
  Pick a system of linear equations from the text and try solving it by substitution.
  Make sure you have plenty of paper.  
  Do you see why this is kind of a bad idea.
\end{UnimpEx}
The other alternative is called Gaussian Elimination or Row Reduction.  
Consider the system
\[\begin{array}{ccccccccc}
  a_{1,1}x_1 & + & \cdots & + & a_{1,n}x_n & - & b_1 & = 0\\
  \vdots     &   & \ddots &   & \vdots     &   & \vdots\\
  a_{m,1}x_1 & + & \cdots & + & a_{m,n}x_n & - & b_m & = 0
\end{array}\]
We will develop a group of very simple operations, each of which does not change the solution set.  
The idea is this:
if you know modifying the system in a certain way will not change the solution set, you are of course free to do that modification.
If that modification makes the system simpler, then you have made progress.  
It is merely a generalization of the very familiar principle that if you see ``$x-4=5$'', you want to ``subtract 4 from both sides'', because you know that operation does not change the solution, but will make the equation much simpler.  

The first operation is just swapping the position of two equations.  
Next, is multiplying through by a scalar, that is, if you had an equation
\[a_{i,1}x_1 + \cdots + a_{i,n}x_n - b_i  = 0\]
and $c$ is some nonzero scalar, then replace this equation with
\[(ca_{i,1})x_1 + \cdots + (ca_{i,n})x_n - (cb_i)  = 0\]
Finally, you can replace an equation with the sum of two other equations, that is, given two equations from the system
\[\begin{array}{ccccccccc}
  a_{i,1}x_1 & + & \cdots & + & a_{i,n}x_n & - & b_i & = 0\\
  a_{j,1}x_1 & + & \cdots & + & a_{j,n}x_n & - & b_j & = 0
\end{array}\]
replace one of them with
\[\begin{array}{ccccccccc}
  (a_{i,1} + a_{j,1})x_1 & + & \cdots & + & (a_{i,n} + a_{j,n})x_n & - & (b_i + b_j) & = 0
\end{array}\]

\begin{Ex}
  \label{sec:gaussElimRules}
  Verify that the following three operation do not change the solution set.  
  \begin{enumerate}[a)]
  \item Swapping two equations
  \item Multiplying an equation by a non-zero scalar
  \item Replacing an equation with the sum of two equations
  \end{enumerate}
  Your answer should be of the form ``If $x_1,...,x_n$ is some solution, and we do an operation, then the resulting system of equations has the same solution set, because $<$fill in the blank$>$''.  Notice how the sentence above does not depend on you knowing what that solution is, or even that one exists.  
\end{Ex}

Thus, if you want to find a solution to a system of equations, you know you won't get the wrong answer by using these operations.  
\begin{ImpEx}
  \label{sec:gaussElim}
  Try out a few examples from the text; pick a system of linear equations and apply these operations until the system is simple enough for you to solve ``by hand''.
  Invent strategies for clearing out variables and see just how simple you can make a system.
  Try to get as many variables as possible to appear in only one equation.
  How do you know when you're ``done''?
\end{ImpEx}



\exersisese

\section{Matrices}

You probably noticed in \ref{sec:gaussElim} a few things.
First of all, you might have noticed that ``leaving a blank spot'' when a variable didn't appear in an equation was useful (if not, think about why it might be).  
For instance, it would be useful to write the system like this:
\[\begin{array}{ccccccccc}
  x_1 & + & 5x_2 & + & -3x_3 & - & 2 & = 0\\
   &  & 2x_2 & &  & - & 5 & = 0
\end{array}\]
rather than
\[\begin{array}{ccccccccc}
  x_1 & + & 5x_2 & + & -3x_3 & - & 2 & = 0\\
   2x_2 & -& 5 & = & 0
\end{array}\]
Lining things up is helpful because it suggests row reduction strategies and allows you to check on your ``progress'' at a glance.

The second thing you might have noticed is that you spend an awful lot of time drawing $x$'s and $+$'s, even though they don't really tell you anything interesting, especially if you aligned your system as mentioned above.
Thus it would be just as good to only write the coefficients, so long as you kept track of the blank spaces by putting a zero.
We don't even really have to keep track of the right hand side of the equation, because we know its just a zero.  
Thus we could write the system above ``compactly'' as
\[\left(\begin{array}{cccc}
  1 & 5 & -3 &  2\\
  0 & 2 & 0  &  5
\end{array}\right)\]

We define a matrix just like we did a vector
\begin{Def}[Matrix]
  An $m\times n$ \emph{matrix} is a rectangular array of real numbers with $n$ columns and $m$ rows.  
  If $A$ is an $m\times n$ matrix, we denote the entry in column $i$ and row $j$ by $A_{i,j}$.
\end{Def}

\begin{Remark}
  It is important to remember this is just a syntactic transformation: just an easier way of writing down a system of equations.
  It means the exact same thing, although we hope this format can yield insight about the problem at hand.   
\end{Remark}

\begin{Remark}
  Everybody gets the order of the entries confused. 
  I often write down a matrix you wonder if it should be $m\times n$ or $n\times m$, I find this horribly confusing (see \ref{sec:confusing}).
  Just always remember $n$ is the number of variables and figure it out from there.  
\end{Remark}

\begin{RemarkProg}
  This translates quite literally to a programming paradigm: store the ``system of an equation'' in a 2-d array.  
  This is good because you don't need to waste memory on the $x_i$'s, has good locality and if you want to know ``the coefficient of $x_3$ in equation 2'', you just look at $A[2][3]$. 
  Interestingly, if space is limited and you have a large system where each equation only has a few variables (called a sparse system), you end up wasting a lot of space on zeros, so a ``literal'' format might be preferred.  
\end{RemarkProg}

We might wish to remind ourselves that the $b_i$ special, in the sense that they are not to be eventually multiplied by an $x_i$.  
This leads to the concept of an ``augmented matrix'', which is a purely syntactic concept designed to keep us from getting confused while working with matrices.  

\begin{Def}[Augmented Matrix]
  We will often draw a vertical dashed line separating the $b_i$'s column from the rest of the matrix, just to visually remind elf's what the matrix represents.  
  We call such a \emph{matrix augmented}, and the column to the right of the line the augmented column.  
\end{Def}

I find Gaussian Elimination is easier to think about with matrices, since the operations are just row-wise additions, multiplications and swaps.  
\begin{Ex}
  What do the Gaussian Elimination rules (\ref{sec:gaussElimRules}) look like in matrix form?  
\end{Ex}

In \ref{sec:gaussElim}, I asked how do you know when you're done?  
With the system written as a matrix, this is an easy-to-check.   
You are done when your matrix is in ``reduced row echelon form'', or ``rref''.  
That is a scary sounding name (I have no idea with echelon means), but it just means the following.
\begin{Def}[Reduced Row Echelon Form]
  An $m\times n$ matrix $A$ is in \emph{reduced row echelon form} if
  \begin{enumerate}
  \item Each column either contains only a single 1 (and the rest zeros), or the nonzero entries are \emph{not} the row's leftmost nonzero entry.
  \item The rows are sorted by their first nonzero entry.
  \end{enumerate}
  We call columns with only a single one \emph{pivot columns} and otherwise \emph{free columns}.  
\end{Def}

\begin{Def}[Gaussian Elimination]
  Gaussian Elimination is the process of using cleverly chosen row operations until a matrix is in reduced row echelon form.  
\end{Def}

\noindent Why is reduced row echelon form so nice?  

\begin{Example}
  \label{sec:gaussexamp}
  Consider this rref'd matrix.
  \[\left(\begin{array}{cccc}
      1 & 0 & -3 &  2\\
      0 & 1 & 1  &  \frac{1}{2}\\
      0 & 0 & 0 & 0
    \end{array}\right)\]
  Remember that a matrix is just shorthand for a system of equations
  \[\begin{array}{cccccccc}
    x_1 & + & & & -3x_3 & - &  2 & = 0\\
    & & x_2& + & x_3 & - &  \frac{1}{2} & = 0\\
    & & & &&  &  0 & = 0\\
  \end{array}\]
  Notice that you can set $x_3$ to anything you want and you automatically know what $x_1$ and $x_2$ are; that is, we can rearrange the system to look like.
  \[\begin{array}{cccccccc}
    x_1 & = & 3x_3 & + &  2\\
    x_2& = & -x_3 & + &  \frac{1}{2}\\
    0 &=&0
  \end{array}\]
  Thus for each possible value of $x_3$ there is exactly one solution to the system.  
  This is about as simple a system of equations as you can ask for, as it has the property that you can ``just plug in any value for $x_3$ and get a solution''.  
  However, we can actually write this in a slightly ``slicker'' form.
  We know that the solution to the system is a vector 
  \[\vec{x}=\vect{x_1 \\ x_2 \\ x_3}\]
  As we saw before, we can write $x_1$ and $x_2$ in terms of $x_3$.  
  In a completely trivial way, (the kind of way which is so trivial you'd never think of it), we can ``write $x_3$ in terms of $x_3$''.  
  How?  $x_3=x_3$ (womp womp).  
  Thus we can write this as
  \[\vec{x}=\vect{x_1 \\ x_2 \\ x_2}= \vect{3x_3+2 \\ -x_3+\frac{1}{2} \\ x_3}\]
  We can do better still.  
  Each of the coordinates of the right hand vector are of the form $ax_3+b$, where $b=0$ in the last coordinate.  
  Thus we can split this vector up as a sum, and then factor out the $x_3$, like so:
  \[\vec{x}=\vect{x_1 \\ x_2 \\ x_2}= \vect{3x_3+2 \\ -x_3+\frac{1}{2} \\ x_3}
  =\vect{3x_3 \\ -x_3 \\ x_3}+\vect{2 \\ \frac{1}{2} \\ 0} = x_3\vect{3 \\ -1 \\ 1}+\vect{2 \\ \frac{1}{2} \\ 0}\]
  This is very cool!
  Writing the solution this way emphasizes the geometric nature of the solution set, and makes it is obvious that the solutions form a line!
  Do you see why?
\end{Example}

\begin{ImpEx}
  \label{sec:guassexampex}
  Convince yourself the solution set forms a line in $\R^3$.  
  Can we interpret this as ``the line line going through a certain point in a certain direction''?
\end{ImpEx}

\begin{ImpEx}
  \label{sec:gaussexamp2}
  Go through the logic of \ref{sec:gaussexamp} for the following rref'd matrix, and then do \ref{sec:guassexampex} for it.  
  \[\left(\begin{array}{ccccc}
      1 & 0 & -3 & -1 & 2\\
      0 & 1 & 1  &  4 & \frac{1}{2}\\
      0 & 0 & 0 & 0
    \end{array}\right)\]
  What is the geometry of the solution set? (hint: the system is the same except for there is one more free variable)
\end{ImpEx}

\begin{Def}[Parametric Form]
  We call a solution to a system of equations in parametric form if it is written a constant vector plus a sum of variables times constant vectors, as in \ref{sec:gaussexamp} and \ref{sec:gaussexamp2}.  
\end{Def}

\begin{Example}
  Consider the following matrix in rref.  
  \[\left(\begin{array}{cccc}
      1 & 0 & -3 &  0\\
      0 & 1 & 1  &  0\\
      0 & 0 & 0 & 1
    \end{array}\right)\]
  Notice the bottom row has non nonzero entries corresponding to any $x_i$, but a nonzero entry in the $\vec{b}$ column.  
  Thus when we write out the system of equations, we get
    \[\begin{array}{cccccccc}
    x_1 & + & & & -3x_3 &  &   & = 0\\
    & & x_2& + & x_3 &  &   & = 0\\
    & & & &&  &  1 & = 0\\
  \end{array}\]
  We know that $1\ne 0$, so no matter what the $x_i$ are, the bottom equation is false, and thus the whole system is false, so the system has no solutions.  
\end{Example}

\begin{Ex}
  \label{sec:allparam}
  Show that a system of linear equations is inconsistent if and only if the augmented column of the corresponding matrix is a pivot column in the rref form.  
  Additionally, show that if there is a solution (that is, if the system is not inconsistent), we can write it in parametric form.  
  (hint: show the method of parametrization from \ref{sec:gaussexamp} and \ref{sec:gaussexamp2} works whenever the augmented column is not a pivot)
\end{Ex}

\begin{ExProg}[(optional)]
  Try writing a computer program (in the language of your choice) to put a matrix in rref. 
\end{ExProg}
We end with an easy definition:
\begin{Def}{Matrix Addition}
  We add two matrices of the same size by adding corresponding entries, that is
  \[\left(\begin{array}{cccc}
      a_{1,1} &  \cdots & a_{1,n}\\
      \vdots     & \ddots  & \vdots\\
      a_{m,1} & \cdots  & a_{m,n}
    \end{array}\right) + 
  \left(\begin{array}{cccc}
      b_{1,1} &  \cdots & b_{1,n}\\
      \vdots     & \ddots  & \vdots\\
      b_{m,1} & \cdots  & b_{m,n}
    \end{array}\right) =
  \left(\begin{array}{cccc}
      a_{1,1} + b_{1,1} &  \cdots & a_{1,n} + b_{1,n}\\
      \vdots     & \ddots  & \vdots\\
      a_{m,1}+a_{m,1} & \cdots  & a_{m,n}+a_{m,n}
    \end{array}\right)\]
  We define the zero matrix as the matrix with all zero entries.  
\end{Def}


\exersisesf

\section{Matrix Vector Products}

Recall that a homogeneous linear equation is a one where the augmented column is all zeros.  
\begin{EasyEx}
  Show that if a system of linear equations is homogeneous, then the rref'd form of the matrix for that system is homogeneous.  
  (hint: what do each of the three operations of Gaussian Elimination do to the augmented column of such a matrix?)
\end{EasyEx}
Since throughout our analysis the last column stays zero, we may neglect it and represent our system of equations by a non-augmented matrix.  
A homogeneous system of equation looks like this:
\[\begin{array}{ccccccc}
  a_{1,1}x_1 & + & \cdots & + & a_{1,n}x_n  & = 0\\
  \vdots     &   & \ddots &   & \vdots     \\
  a_{m,1}x_1 & + & \cdots & + & a_{m,n}x_n & = 0
\end{array}\]
The corresponding matrix, which we will call $A$, is
\[A=\left(\begin{array}{cccc}
  a_{1,1} &  \cdots & a_{1,n}\\
  \vdots     & \ddots  & \vdots\\
  a_{m,1} & \cdots  & a_{m,n}
\end{array}\right)\]
Define $\vec{a_i}$ to be the vector with $j$th coordinate $a_{i,j}$, that is, $\vec{a_i}$ is the $i$th row of $A$.
Then recall, as in \ref{sec:dot-is-lineareq}, that our system of equations can also be written
\[\begin{array}{ccc}
  \vec{a_1}\cdot\vec{x} & = & 0\\
  \hdots &&\hdots\\
  \vec{a_n}\cdot\vec{x} & = & 0
\end{array}{ccc}\]
Following the philosophy of atomizing things, it would be nice if we could write this system of many equations as some kind of single equation involving vectors.
To do this though, we need to make a definitions.
\begin{Def}[Matrix Vector Product]
  Let $A$ be an $m\times n$ matrix with rows $\vec{a_i}\in\R^m$.
  If $\vec{x}\in\R^n$, then define the matrix-vector product
  \[A\vec{x} = \left(\begin{array}{cccc}
  a_{1,1} &  \cdots & a_{1,n}\\
  \vdots     & \ddots  & \vdots\\
  a_{m,1} & \cdots  & a_{m,n}
\end{array}\right)\vect{x_1\\ \hdots\\x_n} = 
\vect{\vec{a_1}\cdot\vec{x}\\
  \hdots\\
  \vec{a_n}\cdot\vec{x}}\]
\end{Def}
\begin{Remark}
  With each new operation it is important to remember the type.
  In this case: 
  \[\mbox{matrix times vector = vector}\]
  Or even better
  \[\mbox{$m\times n$-matrix times $n$-vector = $m$-vector}\]
\end{Remark}
\begin{PedRemark}
  Students often find this definition unmotivated, as if it came out of left field.
  It is extremely important that you understand why the product is defined the way it is, and why we want a matrix vector product at all!
\end{PedRemark}
\begin{EasyEx}
  \label{matvecprodiseq}
  Let $A$ be the non-augmented matrix for a system of $m$ homogeneous linear equations in $n$ variables.  
  Show that the system can be written
  \[A\vec{x}=\vec{0}\]
\end{EasyEx}
\begin{EasyEx}
  Given a system of equations:
  \[\begin{array}{ccccccccc}
    a_{1,1}x_1 & + & \cdots & + & a_{1,n}x_n & - & b_1  & = 0\\
    \vdots     &   & \ddots &   & \vdots     \\
    a_{m,1}x_1 & + & \cdots & + & a_{m,n}x_n &  - & b_m= 0
  \end{array}\]
  Let $A$ be the non-augmented matrix associated with this system, that is, 
  \[A=\left(\begin{array}{cccc}
      a_{1,1} &  \cdots & a_{1,n}\\
      \vdots     & \ddots  & \vdots\\
      a_{m,1} & \cdots  & a_{m,n}
    \end{array}\right)\]
  and let
  \[\vec{b} = \vect{b_1\\\hdots\\b_m}\]
  Show that the original system of linear equations is equivalent to the matrix equation
  \[A\vec{x}=\vec{b}\]
  This generalizes \ref{matvecprodiseq}.
\end{EasyEx}

\begin{Ex}
  \label{sec:othermatvecdef}
  Show that the following definition of matrix-vector multiplication is equivalent to the original.
  If $A$ is an $m\times n$ matrix  and $\vec{x}\in\R^n$, 
  \[\left(\begin{array}{cccc}
      a_{1,1} &  \cdots & a_{1,n}\\
      \vdots     & \ddots  & \vdots\\
      a_{m,1} & \cdots  & a_{m,n}
    \end{array}\right)\vect{x_1\\\vdots\\x_n} =
  x_1\vect{a_{1,1}\\\vdots\\a_{m,1}} + \cdots + x_n\vect{a_{1,n}\\\vdots\\a_{m,n}}\]
  In particular, $A\vec{x}$ is a linear combination of the columns with coefficients the coordinates of $\vec{x}$.  
\end{Ex}


\begin{Def}[Identity Matrix]
  \label{sec:iddef}
  Fix some $n$.  
  Let $I$ (sometimes called $I_n$) be an $n\times n$ matrix with 1 on the diagonal entries and 0 elsewhere, that is
  \[I=\left(\begin{array}{cccc}
      1 & 0  & \cdots & 0\\
      0 & 1  & \cdots &  0\\
      \vdots && \ddots  & \vdots\\
      0 &0 & \cdots  & 1
    \end{array}\right)\]
  We will see that this is a very special matrix.  
\end{Def}
\begin{Ex}
  \label{sec:matvecprops}
  Verify the following properties:
  \begin{enumerate}[a)]
    \item $(A+B)\vec{x} = A\vec{x} + B\vec{x}$
    \item $A(\vec{x} + \vec{y}) = A\vec{x} + A\vec{y}$
    \item $A\vec{0} = \vec{0}$
    \item If $Z$ is a $m\times n$ matrix of all zeroes, $Z\vec{0} = \vec{0}$ (the two zero vectors are different... where do they live?)
    \item $I\vec{x}=\vec{x}$ (see \ref{sec:iddef})
  \end{enumerate}
  (hint: recall that each entry of a matrix vector product is a dot product, and that dot products obey similar looking laws)
\end{Ex}
\begin{Ex}
  \label{sec:extractcol}
  If we wish to extract a column from a matrix, there is a nice way of doing so.
  Show that if $A$ is an $m\times n$ matrix and $\vec{e}_i$ is the $i^{th}$ standard basis vector (that is, all zeros except for a 1 in the $i^{th}$ position), then $Ae_i$ is the $i^{th}$ column of $A$.
  Keep this in mind, it can be a handy shortcut come exam time.
\end{Ex}
\begin{Remark}
  It may seem silly at first to define all this notation, but it gives us the advantage of being able to write a system of linear equations in a few short symbols.  
  Also, the symbols $A\vec{x}=\vec{b}$ looks an awful lot like the very simple scalar equation $ax=b$.  
  In fact, if $m=n=1$, it is exactly that.  
  We will see in \ref{sec:lineartransforms} that ``multiplying by a matrix'' is actually a quite reasonable generalization of ``multiplying by a number'' to situations with many variables.
\end{Remark}
\exersisesg

\end{document}