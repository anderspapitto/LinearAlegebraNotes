\section*{Preface}
\addcontentsline{toc}{section}{Preface}

\subsection*{Who This Book is For and What this Book is About}
\addcontentsline{toc}{subsection}{Who This Book is For and What this Book is About}


This book is for any math 51 student (or student of a comparable linear algebra course at another university).
The target reader is someone who is not necessarily a huge math geek, but wants to do well in this course and is willing to put in a reasonable amount of work. 
This book is designed to help focus such a student's learning, fill in gaps left by the regular text book and ask the kind of questions that will stimulate the reader's understanding of linear algebra.
Translation: I want to help make yourself battle-ready for exams and future courses and professional work (but secretly I want to make you think this is as cool as I do).  

This book is meant to \emph{supplement} the lectures and course text, not replace it.  
My hope is to explain the material in a clear way, emphasizing the connections between different parts of the book and the reasons we think linear algebra is \emph{so cool}.
I have tried to err on the side of being more verbose, since the course text is often rather terse, and I try to motivate each section with a problem or curiosity.
Much of this text is devoted to conceptual exercises, as a sort of way of ``Socratic Dialog''.  
The purpose of these exercises is two-fold.
The first purpose is the exams.
The math 51 exams are notoriously different from the problem sets.
The problem sets emphasize mechanically working with vectors and matrices, while the exams tend to emphasize conceptual understanding and synthesis.
Thus if the only problems you do are on the problem sets, you will get really good at row-reducing matrices, but less good at using the ideas in new situations.  
The second purpose is that it is a well known fact that if you discover something for yourself, you will learn it better, so I attempt to ask you the right questions so you can really get this material into your soul.

The reader shouldn't need a whole lot of formal math knowledge or skill.
I try to, as much as possible, ``de-formalize'' the math.
The formal proofs, to me, are less important than the ideas and the connections between them.

I have been tutoring math 51 for over three years and using linear algebra in advanced mathematics and computer science course work and as an software engineer.  
I know this material like the back of my hand, and I've worked with such a huge number of diverse students that I feel like know where students get confused, and how students succeed.  
I think all that experience will help.

\subsection*{How to Use This Book}
\addcontentsline{toc}{subsection}{How to Use This Book}

The best way of using this book would be to read the section after you go over it in class, making sure you can do \emph{all} the exercises (especially the exam ones).
The second best way to use this book is just to do the exercises from the section after you go over it in class, and only reading my blah-blah-blah expositions if you are having trouble.
The third best way to use this book is just to, the week before the exam, go to the end of each section and do the ``Exam Exercises'', and then if you can't get certain exam questions, try reading the chapter.

Each chapter has an ``Exam Exercises'' section at the end with links to questions from the actual exams.
This is \emph{very important}!
The exams are worth the vast majority of your grade in this class, but most students don't do practice exams until the week before the exam.
This is actually an entirely reasonable thing to do, since if you just cracked open a practice exam a few weeks early, chances are, you wouldn't be able to do half of it since you haven't learned all the material yet.
Even when the exam is coming up, just doing practice exams (though a great thing to do) can be inefficient, for several reasons.
First of all, if you can't do a problem, you might not have any clue what skills/chapters you need to brush up on.
Second of all, if you can't do a problem because you haven't learned skill $X$, and $X$ depends on skill $Y$ (which you haven't quite mastered either), you might waste a lot of time being confused about $X$ when if you spent that time working on $Y$, you would be able to make progress and would feel much better about yourself.
Third of all, its just plain intimidating.
My hope organizing these exam questions in these ways will be more manageable, focused, and allow you to start working on skills which actually matter for your grade a little bit every night, rather than all in one week.

\[\mbox{If all you do is try practice exam problems as you are learning the material, I think you will succeed.}\]

\subsection*{A Word Exercises}
\addcontentsline{toc}{subsection}{A Word Exercises}

For the most part, the exercises I provide should be learning experiences.
I usually ask you to show or realize something.
The thing I ask you to do will usually turn out to be a useful tool in your arsenal.
In fact, most of the real ``content'' in this book is meant to be \emph{discovered} through exercises, and many of the exercises rely on tools learned during other exercises.
If you'd like, pretend you're playing an RPG game and you want to expand your character's skills and experience.
To maximize your success in a given chapter, make sure you know how to do the exercises from the chapter before!

You shouldn't worry all too much about the formal mechanics of proofs.
If I say to show something, I really just mean convince yourself it must be true.  
If you aren't sure whether your reasoning makes sense, try explaining it to a class mate, a dog or a tutor.
It really only matters that you understand, so I try to ``de-formalize'' things by removing the logical ``code words''; you shouldn't be tripping over set builders, ``for-alls'', ``there exists'', etc.
The only time in this course when some logical chops might be required is in proving that things are subspaces, so a sort of template is provided.  

There are several types of exercises in this book, which is a concept I stole from Ravi Vakil's awesome Algebraic Geometry book.
Let me tell you about them
\begin{enumerate}[$\bullet$]
  \item Easy Exercises

    These should be very very short, and seem ``obvious'' after you do them.  
    Usually I just want you to make a simple observation, make an easy connection, or write down the definition of something in a specific case and notice something about it.
    There is nothing, usually, to write down during these exercises, or if there is it will be some small thing in the margin.
    They are things that the textbook might take for granted and the lecturer might go over pretty fast because its so simple, but it is important that you come to the conclusions on their own.
    You shouldn't feel bad if one of these exercises takes a little while to do; time spent pondering now will make these observations second-nature when it counts.
    Just because the solution is small and ``obvious-in-retrospect'' doesn't mean it will be obvious the first time!

  \item Important Exercises
    
    These exercises are either to understand something very important which other things rest on, or to use a skill which is very important and other things rest on.
    If you are limited in time, do these.
    Again, like the easy exercises, they might take a little while to work out the first time, but after you do you will have internalized something that might come in useful later, and if you try to do them again it should be quick.

  \item Tricky Exercises

    These are usually optional, as they can take a bit longer and some more insights, but you should still try to do them.
\end{enumerate}
    
\subsection*{Other Resources}
\addcontentsline{toc}{subsection}{Other Resources}

Books are great because you can work with them any time and at any pace, but nothing can replace working with a person one-on-one.
In addition to taking advantage of office hours with TA's and professors, you can stop by SUMO hours and work with the tutors.
I was a SUMO tutor for years and I can tell you that the tutors are knowledgeable and excited to help you with math.
Also, for what its worth, the SUMO tutors are never involved with grading, which makes some people more comfortable asking for help (although I assure you, the TA's and professors would never penalize you for asking for help).  
Anyways, the SUMO sessions and great, so if you want to go you can find the exact time and date on the math 51 website (or probably posted in the hallways).  

\subsection*{Why Don't You have Multivariable Calculus Too?}
\addcontentsline{toc}{subsection}{Why Don't You have Multivariable Calculus Too?}

Because, despite what you might think about somebody who writes math books in their spare time, I 

\subsection*{Where Can I Give Feedback}
\addcontentsline{toc}{subsection}{Where Can I Give Feedback}

\[\mbox{PLEASE PLEASE PLEASE Tell me how I can improve this!}\]

Shoot me an email at jvictor@stanford.edu.  
I would love comments like ``you aren't very clear in this part'', or ``I want more exercises of this type'', or ``this exercise is too easy or too hard'', 
or ``one of your links is broken or points to a question that doesn't fit with the chapter''.
Or just tell me how great this is and how it helped you get an $A+$.  
Or tell me how much it sucks.  
Just tell me what you think and I'll try to improve it.


\subsection*{Contributing}
\addcontentsline{toc}{subsection}{Contributing}

We're on GitHub!
The repo is public!  
Clone me with
\begin{verbatim}
    git clone https://github.com/jvictor0/LinearAlegebraNotes.git
\end{verbatim}

